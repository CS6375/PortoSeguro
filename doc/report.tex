\documentclass[conference,draft]{IEEEtran}

%\usepackage{amsmath}
\usepackage{booktabs} % For formal tables

\usepackage{cite}

\usepackage{blindtext}

\begin{document}
\title{Porto Seguro's Safe Driver Prediction\\Project Status Report}
\author{
\IEEEauthorblockN{Hanlin He, Mingze Xu, Su Yang, Tao Wang}
\IEEEauthorblockA{Department of Computer Science, The Erik Jonsson School of Engineering and Computer Science\\
The University of Texas at Dallas\\
Email:
\{hxh160630, mxx160530, sxy161730, txw162630\}@utdallas.edu}}

\maketitle

\section{Introduction}

%Motivate and abstractly describe the problem you are addressing and how you are addressing it.
%What is the problem?
%Why is it important?
%What is your basic approach?
%A short discussion of how it fits into related work in the area is also desirable.
%Summarize the basic results and conclusions that you will present.

Machine learning is emerging in the insurance industry and is being applied
across multiple areas including the interpretation of data, business operations
and driver safety. One key application is claim prediction. Inaccuracies in car
insurance company's claim predictions raise the cost of insurance for good
drivers and reduce the price for bad ones. A more accurate prediction will
allow insurers to further tailor their prices, and hopefully make auto
insurance coverage more accessible to more drivers.

In this report, we focus on \emph{Porto Seguro's Safe Driver Prediction}\cite{Domingos:2012:FUT:2347736.2347755}.
competition hosted on Kaggle, predicting the possibility that an auto insurance
policy holder will file an insurance claim next year.

The organization of the following report is as follows. In section II, we will
formally define the problem to solve and discuss the theoretical principle of
the algorithm we used. Then we will analyze the data feature and our method of
feature engineering in section III. After that, the experimental results are
shown and analyzed in section IV. Finally, we will discuss related works and
conclude the report.

\section{Problem Definition and Algorithm}

\subsection{Task Definition}

Precisely define the problem you are addressing (i.e. formally specify the
inputs and outputs). Elaborate on why this is an interesting and important
problem. 

\subsection{Algorithm Definition}

Describe in reasonable detail the algorithm you are using to address this
problem. A psuedocode description of the algorithm you are using is frequently
useful. Trace through a concrete example, showing how your algorithm processes
this example. The example should be complex enough to illustrate all of the
important aspects of the problem but simple enough to be easily understood. If
possible, an intuitively meaningful example is better than one with meaningless
symbols. 

\section{Feature Analysis and Engineering}

Feature Engineering.

\section{Experimental Evaluation}

\subsection{Methodology}

What are criteria you are using to evaluate your method? What specific
hypotheses does your experiment test? Describe the experimental methodology
that you used. What are the dependent and independent variables? What is the
training/test data that was used, and why is it realistic or interesting?
Exactly what performance data did you collect and how are you presenting and
analyzing it? Comparisons to competing methods that address the same problem
are particularly useful. 

\subsection{Results}

Present the quantitative results of your experiments. Graphical data
presentation such as graphs and histograms are frequently better than tables.
What are the basic differences revealed in the data. Are they statistically
significant? 

\subsection{Discussion}

Is your hypothesis supported? What conclusions do the results support about the
strengths and weaknesses of your method compared to other methods? How can the
results be explained in terms of the underlying properties of the algorithm
and/or the data. 

\section{Related Work}

Answer the following questions for each piece of related work that addresses
the same or a similar problem. What is their problem and method? How is your
problem and method different? Why is your problem and method better? 

\section{Future Work}

What are the major shortcomings of your current method? For each shortcoming,
propose additions or enhancements that would help overcome it. 

\section{Conclusion}

Briefly summarize the important results and conclusions presented in the paper.
What are the most important points illustrated by your work? How will your
results improve future research and applications in the area? 

\section{Bilbiography}

Be sure to include a standard, well-formated, comprehensive bibliography with
citations from the text referring to previously published papers in the
scientific literature that you utilized or are related to your work.

\bibliography{report.bib}{}
\bibliographystyle{plain}

\end{document}
