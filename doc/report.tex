\documentclass[conference,draft]{IEEEtran}

%\usepackage{amsmath}
\usepackage{booktabs} % For formal tables

\usepackage{cite}

\usepackage{listings}
\lstset{ %
    basicstyle=\footnotesize\ttfamily,
}
\usepackage{hyperref}

\usepackage{blindtext}

\begin{document}
\title{Porto Seguro's Safe Driver Prediction\\Project Status Report}
\author{
\IEEEauthorblockN{Hanlin He, Mingze Xu, Su Yang, Tao Wang}
\IEEEauthorblockA{Department of Computer Science, The Erik Jonsson School of Engineering and Computer Science\\
The University of Texas at Dallas\\
Email:
\{hxh160630, mxx160530, sxy161730, txw162630\}@utdallas.edu}}

\maketitle

\section{Introduction}

%Motivate and abstractly describe the problem you are addressing and how you are addressing it.
%What is the problem?
%Why is it important?
%What is your basic approach?
%A short discussion of how it fits into related work in the area is also desirable.
%Summarize the basic results and conclusions that you will present.

Machine learning is emerging in the insurance industry and is being applied
across multiple areas including the interpretation of data, business operations
and driver safety. One key application is claim prediction. Inaccuracies in car
insurance company's claim predictions raise the cost of insurance for good
drivers and reduce the price for bad ones. A more accurate prediction will
allow insurers to further tailor their prices, and hopefully make auto
insurance coverage more accessible to more drivers.

In this report, we based on Kaggle's Featured Prediction Competition
\emph{Porto Seguro's Safe Driver Prediction}\cite{kaggle}, conducted several
experiments, compared different approaches' effectiveness to tackle the claim
prediction problem, including \emph{logistic regression}, \emph{tensorflow
estimator} and \emph{random forest}.

The organization of the following report is as follows. In section II, we will
formally define the problem to solve and discuss the theoretical principle of
the algorithm we used. Then we will analyze the data feature and our method of
feature engineering in section III. After that, the experimental results are
shown and analyzed in section IV. Finally, we will discuss related works and
conclude the report.

\section{Problem Definition and Algorithm}

\subsection{Task Definition}

%Precisely define the problem you are addressing (i.e. formally specify the
%inputs and outputs). Elaborate on why this is an interesting and important
%problem. 

A machine learning problem is defined as to learn from experience $E$ with
respect to some class of tasks $T$ and performance measure $P$, if its
performance at tasks in $T$, as measured by $P$, improves with experience
$E$\cite{Mitchell:1997:ML:541177}.
The claim prediction problem can be defined as follow:
\begin{itemize}[\IEEEsetlabelwidth{Z}] 
    \item[$E$] previous year's policy holders' information and whether or not a
        claim was filed for that policy holder.
    \item[$T$] predicting the possibility that an auto insurance policy
        holder will file an insurance claim next year.
    \item[$P$] the accuracies and Gini Coefficient was used to measure the
        effectiveness of models.
\end{itemize}
\subsection{Algorithm Definition}

\scriptsize{
Describe in reasonable detail the algorithm you are using to address this
problem. A psuedocode description of the algorithm you are using is frequently
useful. Trace through a concrete example, showing how your algorithm processes
this example. The example should be complex enough to illustrate all of the
important aspects of the problem but simple enough to be easily understood. If
possible, an intuitively meaningful example is better than one with meaningless
symbols.
}\normalsize

\subsubsection{Logistic Regression}

Logistic Regression is an approach to learning functions of the form $f:X\rightarrow Y$\cite{Mitchell:2016} or in our case $P(Y|X)$ where $Y$ is discrete-valued, and $X = \langle X_1 ...X_n\rangle$ is any vector containing discrete and continuous variables. The parametric model assumed by Logistic Regression in the case where Y is boolean is:
\begin{IEEEeqnarray}{Rl} 
P(Y=1|X)&=\frac{1}{1+\exp(w_0+\sum_{i=1}^nw_iX_i)}\IEEEnonumber\\
P(Y=0|X)&=\frac{\exp(w_0+\sum_{i=1}^nw_iX_i)}{1+\exp(w_0+\sum_{i=1}^nw_iX_i)}\IEEEnonumber
\end{IEEEeqnarray}

One reasonable approach to training Logistic Regression is to choose parameter values that maximize the conditional data likelihood. We also used regularization to reduce the overfitting problem. The penalized log likelihood function is as followed:
\begin{IEEEeqnarray}{C} 
W \leftarrow \arg\max_W\sum_l\ln{P(Y^l|X^l,W)}-\frac{\lambda}{2}||W||^2\IEEEnonumber
\end{IEEEeqnarray}
where the last term is a penalty proportional to the squared magnitude of $W$.

In general, the algorithm used gradient ascent to repeatedly update the weights in the direction of the gradient, on each iteration changing every weight $w_i$, beginning with initial weights of zero, according to:
\begin{IEEEeqnarray}{C} 
w_i \leftarrow w_i+\eta\sum_lX_i^l(Y^l-\hat{P}(Y^l=1|X^l,W))-\eta\lambda{w_i}\IEEEnonumber
\end{IEEEeqnarray}
where $\eta$ is a small constant which determines the step size. The actual implementation of scikit learn library includes multiple solvers, such as Stochastic Average Gradient (SAG) descent, SAGA and Broyden-Fletcher-Goldfarb-Shanno (LBFGS).

\section{Feature Analysis and Engineering}

The data comes in the traditional Kaggle form of one training and test file
each: \lstinline{train.csv} and \lstinline{test.csv}. Each row corresponds to a
specific policy holder and the columns describe their features. The target
variable is named \lstinline{target} here and it indicates whether this policy
holder made an insurance claim in the past.

Feature Engineering.

\section{Experimental Evaluation}

\subsection{Methodology}

\scriptsize{
What are criteria you are using to evaluate your method? What specific
hypotheses does your experiment test? Describe the experimental methodology
that you used. What are the dependent and independent variables? What is the
training/test data that was used, and why is it realistic or interesting?
Exactly what performance data did you collect and how are you presenting and
analyzing it? Comparisons to competing methods that address the same problem
are particularly useful. 
}\normalsize

\subsection{Results}

\scriptsize{
Present the quantitative results of your experiments. Graphical data
presentation such as graphs and histograms are frequently better than tables.
What are the basic differences revealed in the data. Are they statistically
significant? 
}\normalsize

\subsection{Discussion}

\scriptsize{
Is your hypothesis supported? What conclusions do the results support about the
strengths and weaknesses of your method compared to other methods? How can the
results be explained in terms of the underlying properties of the algorithm
and/or the data. 
}\normalsize

\section{Related Work}

\scriptsize{
Answer the following questions for each piece of related work that addresses
the same or a similar problem. What is their problem and method? How is your
problem and method different? Why is your problem and method better? 
}\normalsize

\section{Future Work}

\scriptsize{
What are the major shortcomings of your current method? For each shortcoming,
propose additions or enhancements that would help overcome it. 
}\normalsize

\section{Conclusion}

\scriptsize{
Briefly summarize the important results and conclusions presented in the paper.
What are the most important points illustrated by your work? How will your
results improve future research and applications in the area? 
}\normalsize

\section{Bilbiography}

\scriptsize{
Be sure to include a standard, well-formated, comprehensive bibliography with
citations from the text referring to previously published papers in the
scientific literature that you utilized or are related to your work.
}\normalsize

\bibliography{report.bib}{}
\bibliographystyle{plain}

\end{document}
