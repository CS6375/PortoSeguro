
\documentclass{standalone}

\begin{document}

\section{Related Work}\label{related}

Machine learning techniques have been widely used in the field of car insurance
to help the insurance companies in predicting the customers' choices in order
to provide more competitive services. Related works concerned following topics.

\subsection{Insurance Purchase Prediction}

Asma S Alshamsi \cite{alshamsi2014predicting} built a classification model
based on random forest that could be applied in predicting which of the
insurance policies would likely to be chosen by the customers. Saba Arslan Shah
and Mehreen Saeed \cite{shahpredicting} gave an overview of the methodology
developed for predicting the purchased policy for a customer (in a Kaggle
hosted competition \cite{kaggle:allstatepurchase}), using logistic regression,
naive Bayes, SVM and random forest.

\subsection{Drive Style Modeling}

\cite{dong2016characterizing}, \cite{nikulin2016driving} analyzed `good' and
`bad' drivers' driving style using unsupervised learning and deep learning with
vehicle sensor data, e.g., GPS. \cite{singhusing} used Convolutional Neural
Networks (CNNs) to perform deep learning on an image dataset  of drivers in
their cars performing specific actions, trying to identify the \emph{unsafe}
actions of a driver.

\subsection{Claim Prediction}

Dal Pozzolo et al. \cite{dal2010comparison} surveyed many popular machine
learning models' performances on claim prediction. The technique and measure
metric (Gini index) used in their paper were very similar to ours, and greatly
inspired our progress. \cite{huangfu2015data} gave a full report about yet
another Kaggle hosted competition Allstate Claim Prediction
Challenge\cite{kaggle:allstate}.

\end{document}
