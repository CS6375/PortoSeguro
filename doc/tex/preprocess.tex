\documentclass{standalone}

\begin{document}

\section{Feature Analysis and Engineering}

The data comes in the traditional Kaggle form of one training and test file
each: \lstinline{train.csv} and \lstinline{test.csv}. Each row corresponds to a
specific policy holder and the columns describe their features. The target
variable is named \lstinline{target} here and it indicates whether this policy
holder made an insurance claim in the past.

\subsection{Data Overview}

In the train and test data, features that belong to similar groupings are
tagged as such in the feature names (e.g., \lstinline{ind}, \lstinline{reg},
\lstinline{car}, \lstinline{calc}). In addition, feature names include the
postfix \lstinline{bin} to indicate binary features and \lstinline{cat} to
indicate categorical features. Features without these designations are either
continuous or ordinal.

\begin{table}[!h]
\renewcommand{\arraystretch}{1.3}
\caption{Feature Counts in Each Category and Type}\label{table_example}
\centering
\begin{tabular}{c|ccc}
\hline
\bfseries Type & \bfseries  Binary & \bfseries  Categorical & \bfseries  Continuous / Oridinal \\ \hline
\ttfamily ind & 11 & 3 & 4 \\ \hline
reg & & &  \\ \hline
car & & &  \\ \hline
calc & & &  \\ \hline
\end{tabular}

\begin{tabular}{c|cccc}
\hline
Type & \bfseries ind & \bfseries reg & car & calc \\ \hline
Binary & & & & \\ \hline
Categorical & & & & \\ \hline
Continuous & & & & \\ \hline
Ordinal & & & & \\ \hline
\end{tabular}
\end{table}

Although feature's categories are provided, the meaning of each feature remains 

\subsection{Missing Value Mechanism and Data Imputation}

Values of -1 indicate that the feature was missing from
the observation.



\end{document}
