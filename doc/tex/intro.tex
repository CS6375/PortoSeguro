\documentclass{standalone}

\begin{document}

\section{Introduction}

Machine learning is emerging in the insurance industry and is being applied
across multiple areas including the interpretation of data, business operations
and driver safety. One key application is claim prediction. Inaccuracies in car
insurance company's claim predictions raise the cost of insurance for good
drivers and reduce the price for bad ones. A more accurate prediction will
allow insurers to further tailor their prices, and hopefully make auto
insurance coverage more accessible to more drivers.

In this report, we based on Kaggle's Featured Prediction Competition
\emph{Porto Seguro's Safe Driver Prediction}\cite{kaggle}, conducted several
experiments, compared different approaches' effectiveness to tackle the claim
prediction problem, including \emph{logistic regression}, \emph{random forest}
and \emph{gradient boosting}.

The organization of the following report is as follows. In \cref{problem}, we
will formally define the problem to solve and discuss the theoretical principle
of the algorithm we used. Then we will analyze the data feature and our method
of feature engineering in \cref{preprocessing}. After that, the experimental
results are shown and analyzed in \cref{evam}. Finally, we will discuss related
works in \cref{related} and conclude the report in \cref{conclusion}.

\end{document}
